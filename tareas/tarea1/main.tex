\documentclass[12pt]{article}

% --- Página y tipografía ---
\usepackage[letterpaper,margin=2.5cm]{geometry}
\usepackage[T1]{fontenc}
\usepackage[utf8]{inputenc} % si compilas con pdfLaTeX
\usepackage{lmodern}
\usepackage{microtype}

% --- Imágenes y color ---
\usepackage{graphicx}
\usepackage{xcolor}

% --- Control fino de espacios ---
\usepackage{setspace}
\setlength{\parindent}{0pt}

\begin{document}
\thispagestyle{empty}

% ===== Encabezado con logos + texto =====
\begin{minipage}[c]{0.18\textwidth}
    \centering
    % Cambia por tu logo izquierdo
    \includegraphics[width=0.95\linewidth]{logo_usac.jpeg}
\end{minipage}
\hfill
\begin{minipage}[c]{0.60\textwidth}
    \small
    Universidad de San Carlos de Guatemala\\
    Escuela de Ciencias Físicas y Matemáticas\\
    Nombre estudiante: Maria de los Angeles Rojas Granizo\\
    Carnet: 202500859 \\
    Programación 1\\
\end{minipage}
\hfill
\begin{minipage}[c]{0.18\textwidth}
    \centering
    % Cambia por tu logo derecho
    \includegraphics[width=1.4\linewidth]{logo_ecfm.jpg}
\end{minipage}

\vspace{0.5cm}

% Línea horizontal superior (gruesa)
\noindent\rule{\textwidth}{1.2pt}

\vspace{0.2cm}

% ===== Título =====
\begin{center}
    {\Large\scshape Tarea 1 Programación}\\[0.3em]
\end{center}

\vspace{0.1cm}

% Fecha
\begin{center}
    \small\scshape 06 de febrero de 2026
\end{center}

\vspace{0.2cm}

% Línea horizontal inferior (gruesa)
\noindent\rule{\textwidth}{1.2pt}

\vspace{0.6cm}

% ===== Caja de resumen =====
\noindent
\colorbox{pink!35}{%
    \parbox{\textwidth}{%
        \vspace{0.6em}
        \textbf{Instrucciones: redacta un breve ensayo en el que describas cuál sería tu área de investigación en Física o Matemáticas y explica por qué la programación te sería útil en dicho campo. }\\[0.3em]
        \small
    }%
}

\vspace{20px}

He descubierto un interés particular por el campo de investigación de \emph{\textbf{la combinatoria y los fundamentos de la matemática}}, ya que me permite comprender cómo se construyen, organizan y relacionan las estructuras matemáticas desde sus bases más esenciales. Esta área, que integra la matemática discreta, la teoría de conjuntos, la lógica matemática y el estudio de redes, ha cambiado mi forma de ver los problemas matemáticos, llevándome a pensar de manera más estructurada, analítica y creativa. \\
Lo que más me atrae de la \emph{\textbf{\underline{combinatoria}}} es la forma en que, a partir de reglas simples, se pueden generar estructuras complejas y resolver problemas que requieren ingenio más que cálculo mecánico. \emph{\textbf{\underline{La lógica matemática}}} ha llamado mi atención debido a que, desde mi perspectiva, fortalece mi capacidad para razonar, argumentar y construir demostraciones sólidas. Por otro lado, el estudio de \emph{\textbf{\underline{redes o grafos}}} me resulta especialmente interesante porque permite modelar relaciones reales, como conexiones, recorridos o flujos, a través de representaciones matemáticas precisas. \\
Mi interés por tomar esta área como línea de investigación también está fuertemente relacionado con la \emph{\textbf{programación}}. En mis años como estudiante he notado que programar no solo consiste en escribir códigos, sino en traducir ideas matemáticas en algoritmos claros y eficientes. Considero que la programación me permitiría experimentar con conceptos de matemática discreta, comprobar resultados, explorar casos particulares y visualizar estructuras que, de otro modo, quedarían únicamente en el plano teórico. \\
Además, muchas de las herramientas fundamentales de la programación —como los algoritmos, las estructuras de datos, la lógica booleana y los grafos— están directamente basadas en los fundamentos matemáticos que estudia esta área. Esta conexión me motiva a profundizar en la investigación, ya que veo en la combinatoria y los fundamentos de la matemática un espacio donde el rigor teórico y la aplicación práctica se complementan de forma natural. \\
En conclusión, me gustaría tomar esta área de investigación porque me permite unir mi interés por el razonamiento lógico, la estructura matemática y la programación. Considero que este campo no solo fortalecerá mi formación académica, sino que también me brindará herramientas para abordar problemas complejos desde una perspectiva analítica, creativa y con múltiples aplicaciones en la ciencia y la tecnología.



\end{document}

\end{document}

